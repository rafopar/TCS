\documentclass[letterpaper,12pt]{article}
%documentclass[superscriptaddress,preprintnumbers,amsmath,amssymb,aps,11pt]{revtex4}
%\usepackage[]{authblk}
%\usepackage{graphics}
\usepackage[dvipdf]{graphics}
%\usepackage{subfig}  % For subfloats
\usepackage{color}
\usepackage[usenames,dvipsnames]{xcolor}
\usepackage{epsfig}
\usepackage{wrapfig}
\usepackage{rotating}
\usepackage{caption}
%\usepackage{subcaption}
\usepackage{subfig}
\usepackage{authblk}
\usepackage{url}

\oddsidemargin = -14mm
\topmargin = -2.9cm6666
\textwidth = 19cm
\textheight = 24cm

\def \rarr {\rightarrow}
\def \grinp {\includegraphics}
\def \tw {\textwidth}
\def\dfrac#1#2{\displaystyle{{#1}\over{#2}}}
\def \dstl {\displaystyle}
\definecolor{GREEN}{rgb}{0.,0.8,0}
\definecolor{RED}{rgb}{1,0,0}
\definecolor{ORANGE}{rgb}{1,0.5,0}

\title{MC simulations}

\begin{document}
 
 This document describes steps, that are used in the Monte Carlo (MC) simulation chain for the TCS and J/$\psi$ analysis.
 The whole MC chain consist of multiple steps: \newline Generation of events (output is hbook file) $\rarr$ hbook2root  $\rarr$ Convert to LUND $\rarr$ GEMC $\rarr$ Decoding $\rarr$ Reconstruction $\rarr$
 Conversion to Root.
  
 \section{GRAPE Event Generator} 
 Events were generated through the event generator called GRAPE ``GRAce-based Monte-Carlo event generator for Proton-Electron collisions'' \cite{GRAPE}.
 This generator has a possibility to generate DIS, quasi-elastic and elastic production of lepton pairs. In our case we are interested in the elastic scattering
 $ep \rarr l^{-}l^{+}p$ where ($l = , e^{\pm},\mu^{\pm}, \tau^{\pm}$). In the case of $e^{-}e^{+}$ pairs, it has possibility to turn on the interference between
 the beam scattered electron and the electron from the lepton pair. The elastic lepton pair production $ep \rarr l^{-}l^{+}p$ is generated using only Bethe-Heitler (BH) diagrams.
 These events will help to understand the acceptance of the CLAS12 detector for the $e^{-}e^{+}p (e^{-})$ final state. In the extraction of angular asymmetries of lepton pairs,
 it is important to have a good understand of the BH process ({\color{Red} Add a reference to a doc for extraction of asymmetries.}).
 \section{Cuts and flags}\label {sec:cuts_flags}
 The conditions of generation is set by the user in the file named ``grape.cards''.
 Detailed descriptions of all flags and variables are shown in \cite{GRAPE}, however below is shown the description of some cuts/flags
 from the grape.cards that is relevant for the E12-12-001 experiment.
 
 \begin{itemize}
  \item EBEAM   $\rarr$   $\mathrm{10600\; MeV}$  //Lepton beam energy
  \item PROCESS	$\rarr$	1 // 1 = elastic process at the proton vertex
  \item LPAIR	$\rarr$	1 // 1 = final state leptons are $e^{-}$ and $e^{+}$. Could be ($\mu^{-},\mu^{+}$) and ($\tau^{-},\tau^{+}$).
  \item ISR 	$\rarr$ 1 // 1 = Initiall state radiation flag is on .
  \item GRASEL  $\rarr$ 2 // 2 = BH proces diagrams are included, including the interference between two electrons in the final state.
  \item Q2RNGME $\rarr$ $0$, $10$ // $Q^{2}$ range in $\mathrm{GeV}$ units at the electron vertex after the ISR.   Effectively no cut
  \item Q2RNGOB $\rarr$ $0$, $10$ // $Q^{2}$ range in $\mathrm{GeV}$ units at the electron vertex before the ISR.   Effectively no cut
  \item MASELL  $\rarr$ $0$, $10^{20}$, $0.6$, $20$ // The invariant mass ranges (int units of $GeV$) of both possible $e^{-}e^{+}$ combinations, 1st two numbers show range for $M^{(1)}(e^{-}e^{+})$ and 2nd two numbers
  show the range for the $M^{(2)}(e^{-}e^{+})$. $M^{(1)}(e^{-}e^{+})$ and $M^{(2)}(e^{-}e^{+})$ satisfy $M^{(1)}(e^{-}e^{+}) < M^{(1)}(e^{-}e^{+})$
  \item Q2P	$\rarr$ $0.01$, $7$ // -$t$ in the units of $\mathrm{GeV}$.
 \end{itemize}

 In addition to above cuts, user can put cuts on all individual final state particle's angles and moment.
%%%%%%%%%%%%%%%%%%%%%%%%%%%%%%%%%%%%%%%%%%%%%%% T A B L E %%%%%%%%%%%%%%%%%%%%%%%%%%%%%%%%%%%%%%%%%%%%%%%%%%%
 \begin{table}[!htb]
 \centering
 \begin{tabular}{|c|c|c|c|c|}
 \hline
		& proton & 1st $e^{-}$ 	& $e^{+}$ & 2nd $e^{-}$ \\ \hline
  THMIN [deg]	& 60	 & 120		& 120	  & 120		\\ \hline
  THMAX [deg]	& 180	 & 180		& 175	  & 180		\\ \hline
  PMIN  [GeV]	& 0	 & 0		& 1	  & 0 		\\ \hline
  PMAX  [GeV]	& 11	 & 11		& 11	  & 11 		\\ \hline
 \end{tabular}
\caption{Angular and moomentum cuts for final state particles.}
  \label{tb:mom_angle_cuts}
 \end{table}
%%%%%%%%%%%%%%%%%%%%%%%%%%%%%%%%%%%%%%%%%%%%%%% T A B L E %%%%%%%%%%%%%%%%%%%%%%%%%%%%%%%%%%%%%%%%%%%%%%%%%%%
 These values are shown in Table \ref{tb:mom_angle_cuts}.
 
 Some of above mentioned cuts might need some clarification.
 
 ``\textbf{Q2RNGME}'' Effectively no cut is placed on variable ``Q2RNGME'', since we are interested here for quasi-real photoproduction $Q^{2} \sim 0$, we haven't placed any cut on the lower bound,
 and since the cross section drops sharply at high $Q^{2}$ values, only very few events will have high $Q^{2}$, which will not add up significantly on the processing time or data volume. 
 
 ``\textbf{MASELL}'' We don't want the to generate invariant mass of $e^{-}e^{+}$ pairs too low, because the the cross-section is very high at low masses, that would
 result most of generated events be unusable, but from another point of view we would like to study the background under at least under $\phi(1020)$ and $\omega(782)$. Because
 of this it was chosen to use the value $0.6\; \mathrm{GeV}$. Also as we don't know, which electron is the one from pair, we put cut only on the $M^{(2)}(e^{-}e^{+})$ (higher one),
 to be more than  $0.6\; \mathrm{GeV}$.
 
 ``\textbf{Q2P}'' Here also the cross section sharply rises up by going to small $-t$, and $0.01\;\mathrm{GeV^{2}}$ is the minimum $-t$ value for producing lepton pair with mass $0.6\; \mathrm{GeV}$.
 
 ``\textbf{Momentum and angular cuts}'' In GRAPE, the positive axis is defined as the proton's momentum direction. Because of this the $0$ angle scattering is actually $180^{\circ}$
 in GRAPE, therefore we have generated with cuts shown in the table, and then all final state particles are rotated by $180^{\circ}$ around the ``Y'' axis.
 The minimum momentum for positrons is $1\; \mathrm{GeV}$, because below that value the CLAS12 acceptance is close to 0., similarly the minimum scattering angle is $175^{\circ}$ ($5^{\circ}$ in Hall-B system).
 Because of interference term between final state electrons, we didn't put any cut on final state electron momentum, and also no cut placed on minimum scattering angle.
 
 \section{Generation and post-generator cuts}
 
 As soon cuts and flags are determined by the user, before generating events, one should run the step called ``integ''. At this step grape calculates the total cross section overt that phase space.
 Depending on cuts this step might take long time, e.g. 10 hours. After this we can generate events.
 The output is produced in a for of hbook files. There are two ntuples inside the hbook, ``h1'' and ``h11''. h1 contains 4 momenta of intial and final state particles, and
 h11 has only one entry, and it has the total integrated cross section over the phase space that was specified by the cuts in ``grape.cards''
 
 As it described in section \ref{sec:cuts_flags} for electron no minimum angle or momentum cut is specified. This means that It is possible to have have two electrons,
 both being out of CLAS12 acceptance. These events are not useful for TCS and $J/\psi$ analysis, and therefore during the conversion of Hbook files to LUND format, and additional
 cut is placed, which is to require the event to have at least one electron witm $P> 1\; \mathrm{GeV}$ and $\theta > 5^{\circ}$.
 
 Next steps are standard: give LUND files as an input to GEMC, then do decoding on the GEMC output, then do reconstruction on decoding output, and finally
 output of reconstruction is converted to Root.
 
 Different steps of the simulation chain are performed in different places. In particular GRAPE works with old version of LINUX and used outdated CERNLIB libraries, and because of this 
 Zhiwen, setup a Virtual Machine (VM) with old RedHat Linux version with working GRAPE \cite{GRAPE_Zhiwen}. The VM is installed in my local laptop, this step is fast within 1 hour about
 15M grape events were generated. Hbook files were transferred to JLab, and converted to Root at JLab. Then Root files were transferred to University of New Hampshire (UNH) computers,
 where all the remaining steps of the simulation chain are performed.
 
 \section{Normalization}
 
 In order to estimate the number of events in the specific phase space of the CLAS12 detector with a given Luminosity $L$. One should use the following formula
 \begin{equation}
  N_{exp} = \frac{\displaystyle \sigma_{Tot}\cdot L\cdot N(\mathrm{recon})}{N_{0}}.
 \end{equation}

 Here, $\sigma_{Tot}$ is the total cross section in the phase space defined by above cuts, $N\mathrm{recon}$ is the number of reconstructed events in a given bin/phase space, $N_{0}$ is the total generated events by grape.
 As an example, let say we use 15 GRAPE generated files, that all are passed through the whole simulation chain. Each files has 500000 generated events,
 then $N_{0}$ will be $50000\times 15$.
 
 \begin{thebibliography}{50}
  \bibitem{GRAPE} \url{http://research.kek.jp/people/tabe/grape/}, \url{https://arxiv.org/abs/hep-ph/0012029}
  \bibitem{GRAPE_Zhiwen} \url{https://jlabsvn.jlab.org/svnroot/solid/evgen/BH/readme}
 \end{thebibliography}

 
\end{document}
