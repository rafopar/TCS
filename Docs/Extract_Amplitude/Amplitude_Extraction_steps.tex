\documentclass[letterpaper,12pt]{article}
%documentclass[superscriptaddress,preprintnumbers,amsmath,amssymb,aps,11pt]{revtex4}
%\usepackage[]{authblk}
%\usepackage{graphics}
\usepackage[dvipdf]{graphics}
%\usepackage{subfig}  % For subfloats
\usepackage{color}
\usepackage[usenames,dvipsnames]{xcolor}
\usepackage{epsfig}
\usepackage{wrapfig}
\usepackage{rotating}
\usepackage{caption}
%\usepackage{subcaption}
\usepackage{subfig}
\usepackage{authblk}
\usepackage{url}

\oddsidemargin = -14mm
\topmargin = -2.9cm
\textwidth = 19cm
\textheight = 24cm

\def \rarr {\rightarrow}
\def \grinp {\includegraphics}
\def \tw {\textwidth}
\def\dfrac#1#2{\displaystyle{{#1}\over{#2}}}
\def \dstl {\displaystyle}
\definecolor{GREEN}{rgb}{0.,0.8,0}
\definecolor{RED}{rgb}{1,0,0}
\definecolor{ORANGE}{rgb}{1,0.5,0}

\title{Path towards the extraction of the scattering amplitude through angular asymmetries in TCS}

\begin{document}
 
 \maketitle
 
 This document describes proposed steps towards the extraction of the real part of the helicity conserving scattering amplitude $\tilde{M}^{--}$ (look \cite{Berger:2001xd} for details) for the Timelike Compton scattering (TCS) process.
 
 
 \begin{equation}
 \sigma(\gamma p \rarr e^{-}e^{+}p) = \sigma_{TCS} + \sigma_{BH} + \sigma_{Int}
 \end{equation}

 
 \begin{thebibliography}{100}
  \bibitem{Berger:2001xd} 
  E.~R.~Berger, M.~Diehl and B.~Pire,
  %``Time - like Compton scattering: Exclusive photoproduction of lepton pairs,''
  Eur.\ Phys.\ J.\ C {\bf 23}, 675 (2002)
  doi:10.1007/s100520200917
  [hep-ph/0110062].
  %%CITATION = doi:10.1007/s100520200917;%%
  %111 citations counted in INSPIRE as of 27 Feb 2019
 \end{thebibliography}

 
\end{document}
