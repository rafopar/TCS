\documentclass[letterpaper,12pt]{article}
%documentclass[superscriptaddress,preprintnumbers,amsmath,amssymb,aps,11pt]{revtex4}
%\usepackage[]{authblk}
%\usepackage{graphics}
\usepackage[dvipdf]{graphics}
%\usepackage{subfig}  % For subfloats
\usepackage{color}
\usepackage[usenames,dvipsnames]{xcolor}
\usepackage{epsfig}
\usepackage{wrapfig}
\usepackage{rotating}
\usepackage{caption}
%\usepackage{subcaption}
\usepackage{subfig}
\usepackage{authblk}
\usepackage{hyperref}
\usepackage{url}
\usepackage{lineno}
\linenumbers

\oddsidemargin = -14mm
\topmargin = -2.9cm
\textwidth = 19cm
\textheight = 24cm

\def \rarr {\rightarrow}
\def \grinp {\includegraphics}
\def \tw {\textwidth}
\def\dfrac#1#2{\displaystyle{{#1}\over{#2}}}
\def \dstl {\displaystyle}
\definecolor{GREEN}{rgb}{0.,0.8,0}
\definecolor{RED}{rgb}{1,0,0}
\definecolor{ORANGE}{rgb}{1,0.5,0}

\title{Path towards the extraction of the scattering amplitude through angular asymmetries in TCS}
\author{Rafayel Paremuzyan}

\begin{document}
 
 \maketitle
 
 This document describes proposed steps towards the extraction of the real part of the helicity conserving scattering amplitude $Re\tilde{M}^{--}$ (look \cite{Berger:2001xd} for details) for the Timelike Compton scattering (TCS) process. Discussions in this note are based on studies which are presented in a workshop at Bochum in Feb 2014 \cite{Bochum_TCS}. At the time of this studies the CLAS12 reconstruction software was not ready, and therefore acceptance and resolution effects were studied using the CLAS12 FASTMC package. It is also discussed here the selection of kinematic points assuming 120 days of beam time with $10^{35}cm^{-2}s^{-1}$ luminosity electron beam. 
 Unfortunately all the codes and data files are lost, therefore almost all plots in this document will be snippets from above mentioned slides \cite{Bochum_TCS}.
 
 \section{Accessing the scattering amplitude} \label{sec:Access_amplitude}
 Experimentally TCS is accessible through $\gamma p \rarr e^{-}e^{+}p$ reaction. Main contributions to this reaction are the Bethe Heitler (BH), TCS and their interference term.
 \begin{equation}
 \sigma(\gamma p \rarr e^{-}e^{+}p) = \sigma_{TCS} + \sigma_{BH} + \sigma_{Int}
 \end{equation}
As discussed in \cite{Berger:2001xd}, the TCS cross-section is very small (Fig. 10 of \cite{Berger:2001xd}) wrt BH cross-section, therefore in the following discussion the TCS part will be neglected. From remaining temrs, BH is well known (calculable within 1\%-2\% precision). The interference term (eq.30 from \cite{Berger:2001xd})
can be represented in a following way.
\begin{equation}
 \sigma_{Int} = a\cdot M^{--} \cdot \frac{L_{0}(\theta)}{L(\theta, \phi)} \cdot cos(\phi)
\end{equation}
where 
\begin{equation}
a = - \frac{\dstl \alpha_{em}^{3}}{\dstl 4\pi s^{2}} \dfrac{1}{-t}\dfrac{M}{Q^{\prime}}
\dfrac{1}{\tau\sqrt{1-\tau}}\dfrac{1 + cos^{2}(\theta)}{sin(\theta)}
\end{equation}
 
As in the paper \cite{Berger:2001xd}, here too, we will use the weighted cross section (WCRS),
$\dfrac{dS}{dQ^{\prime 2} dt d\phi}$, which is obtained from the differential cross section $\dfrac{d\sigma}{dQ^{\prime 2}dt d(cos(\theta))d\phi}$ weighted by $\dfrac{L}{L_{0}}$ and integrated over $\theta$ (in \cite{Berger:2001xd} from $\pi/4$ to $3\pi/4$). The main reason for not integrating over the whole $\theta$ range ($0$ to $\pi$) is because BH dominates over the interference term at $\theta \sim 0^{\circ}$ and $\theta \sim 180^{\circ}$ (the $\dfrac{\sigma_{BH}}{\sigma_{int}}\sim \dfrac{1}{\sin(\theta)}$). The exact values of integration limits is a subject for studies, and these limits must be chosen to minimize uncertainties on the extracted scattering amplitude $M^{--}$.
 
 The WCRS has the following shape
 \begin{equation}
  \dfrac{dS_{Tot}}{dQ^{\prime 2} dt d\phi} = \int_{\pi/4}^{3\pi/4}d\theta \dfrac{d\sigma}{dQ^{\prime 2}dt d(cos(\theta))d\phi} = S_{\mathrm{BH}} + S_{Int} = S_{BH} + A\cdot Re M^{--}\cdot cos(\phi)
  \label{eq:weighted_crs}
 \end{equation}
 Here $A = \dstl \int_{\pi/4}^{3\pi/4}d\theta\cdot a = - \frac{\dstl \alpha_{em}^{3}}{\dstl 4\pi s^{2}} \dfrac{1}{-t}\dfrac{M}{Q^{\prime}}
\dfrac{1}{\tau\sqrt{1-\tau}} \int_{\pi/4}^{3\pi/4}(1 + cos^{2}(\theta))d\theta$

 Now, if one subtracts the weighted BH cross section from the total weighted cross section, then the result will be the weighted interference term which has a cosine dependence on the angle $\phi$.
 With this proposed method we should divide the $\phi\in(0. - 2\pi)$ range into $N_{\phi}$ bins ($N_{\phi}$ to be determined), and for each $\phi$ bin subtract calculated weighted BH cross section from the measured total weighted cross section. The resulting distribution should have a cosine dependence on the angle $\phi$, moreover the amplitude of the modulation is proportional to the scattering amplitude $ReM^{--}$. In order to extract the $ReM^{--}$, this distribution should be fitted with a function
 \begin{equation}
    f = P\cdot cos(\phi)
 \end{equation}

 As an example the plot in Fig.\ref{fig:cos_phi_fit_illustration} is an illustration of a similar fit. Points in the plot
 are not real data points, but they were generated using a Gaussian function with the mean equal to the weighted cross section in a given kinematic bin ($t$, $Q^{\prime 2}$, $\phi$), and sigma equal to the statistical uncertainty (estimated using CLAS12 FASTMC) of the cross section in the given bin.
 %%%%%%%%%%%%%%%%%%%%%%%%%%%%%%%% F I G U R E %%%%%%%%%%%%%%%%%%%%%%%%%%%%%%%%%%%%
\begin{figure}[!htb]
 \centering
 \grinp[width=0.55\tw]{img/An_Example_cos_fit_withtitles.png}
 \caption{ Illustration plot: The $\phi$ angular dependence of the weighted interference term cross section is fitted with cosine function. }
 \label{fig:cos_phi_fit_illustration}
\end{figure}
 %%%%%%%%%%%%%%%%%%%%%%%%%%%%%%%% F I G U R E %%%%%%%%%%%%%%%%%%%%%%%%%%%%%%%%%%%%
 As soon the parameter $P$ will be extracted from the fit, one then can calculate the real part of $M^{--}$
 \begin{equation}
  ReM^{--} = -\frac{\dstl P}{\dstl \frac{\dstl \alpha_{em}^{3}}{\dstl4\pi s^{2}} \cdot \frac{1}{-t}\frac{M}{Q^{\prime}}\frac{1}{\tau\sqrt{1-\tau}}\int_{\pi/4}^{3\pi/4}(1 + cos^{2}(\theta))d\theta }
  \label{eq:Re_M--}
 \end{equation}

 
 \section{The Effect of the CLAS(12) acceptance and the extrapolation}
 It is important to mention that in the above described method method, integration limits over $\theta$, should not be changed for different $\phi$ bins. Varying integration limits, the $\phi$ dependence of the weighted cross section will not have a cosine shape anymore.
 
 The CLAS and CLAS12 detector acceptances have strong $\theta$ and $\phi$ dependence.
 %%%%%%%%%%%%%%%%%%%%%%%%%%%%%%%% F I G U R E %%%%%%%%%%%%%%%%%%%%%%%%%%%%%%%%%%%%
\begin{figure}[!htb]
 \centering
 \grinp[width=0.75\tw]{img/Clas12_Acc.png}
 \caption{ CLAS12 acceptance obtained in different $-t$ bins. Acceptance are calculated
 through CLAS12 FASTMC package.}
 \label{fig:CLAS12_th_phi_acc}
\end{figure}
 %%%%%%%%%%%%%%%%%%%%%%%%%%%%%%%% F I G U R E %%%%%%%%%%%%%%%%%%%%%%%%%%%%%%%%%%%%
 As an example in Fig.\ref{fig:CLAS12_th_phi_acc} shown CLAS12 acceptance obtained in different $t$ bins. One can see that the CLAS12 acceptance doesn't cover the full $\theta$ range, and moreover, the $\theta$ range varies a lot as a function of the angle $\phi$. This means that if in the data we integrate over $\theta$ in the available $\theta$ range, then the resulting distribution will not have a cosine dependence on the angle $\phi$. 
 
 Now let's look into the equation (\ref{eq:weighted_crs}). It shows that the WCRS of
 interference term has as $1 + cos^{2}(\theta)$ dependence on the angle $\theta$ for any angle $\phi$, and if one measures the interference term WCRS in a given range $\theta \in (a,b)$, $^{b}_{a}S_{Int}$, then one can uniquely calculate the interference term WCRS in any $\theta\in(A, B)$ range. 
 \begin{equation}
  \displaystyle ^B_{A}S_{Int} = ^b_{a}S_{Int} \frac{\displaystyle \int_{A}^{B}(1 + cos^{2}(\theta))d\theta}{\displaystyle  \int_{a}^{b}(1 + cos^{2}(\theta))d\theta}
  \label{eq:extrapolation}
 \end{equation}
%Where $^{b}_{a}S_{Tot}$%
%%%%%%%%%%%%%%%%%%%%%%%%%%%%%%%% F I G U R E %%%%%%%%%%%%%%%%%%%%%%%%%%%%%%%%%%%%
The interference term WCRS can be calculated by subtracting the BH WCRS from the total measured WCRS.
\begin{equation}
 ^{b}_{a}S_{Int} = ^{b}_{a}S_{Tot} - ^{b}_{a}S_{BH}
 \label{eq:Int_eq_Tot_minus_BH}
\end{equation}
In short: the $^{b}_{a}S_{Tot}$ will be measured from the experiment, the $^{b}_{a}S_{BH}$ 
will be calculated by integrating the analytic form the BH cross section over the $\theta\in(a-b)$, and therefore the interference term WCRS will be measured using the eq. (\ref{eq:Int_eq_Tot_minus_BH}).
\begin{figure}[!htb]
 \centering
 \grinp[width=0.75\tw]{img/Int_Term_Th_dependence.png}
 \caption{The interference term has a $(1 + cos^{2}(\theta))$ dependence on the angle $\theta$, for any $\phi$ bin. }
 \label{fig:int_th_dep}
\end{figure}
%%%%%%%%%%%%%%%%%%%%%%%%%%%%%%%% F I G U R E %%%%%%%%%%%%%%%%%%%%%%%%%%%%%%%%%%%%
Then using equations (\ref{eq:Int_eq_Tot_minus_BH}) and (\ref{eq:extrapolation}), the interference term WCRS will be calculated for any $\theta\in(a-b)$ range for any given $\phi$ bin.
 As an illustration, in Fig.\ref{fig:int_th_dep} shown the function $(1 + cos^{2}(\theta))$.
 Knowing only the integral of 'blue' shaded area, the integral of the 'Orange' area can be determined using the equation (\ref{eq:extrapolation}). In our case the blue shaded area will correspond the CLAS(12) acceptance, and the orange shaded area will correspond to the $\pi/4$ to $3\pi/4$ range.
 
\section{Estimation of statistical uncertainties}
The statistical uncertainty on the $ReM^{--}$ can be obtained in a following way.
Let's assume the final distribution of the interference term WCRS is the one depicted on Fig.\ref{fig:cos_phi_fit_illustration}. Using this distribution as a source, we will generate a big number (e.g. 10000 or more) of similar distributions, in a way that a WCRS in a given $\phi$ bin will be a Gaussian random number with a Mean equal to the WCRS and sigma equal to the statistical error of corresponding bin of the source distribution. Then each of generated distribution will be fitted with a cosine function in a same way as the source distribution, and each fit will yield a certain $ReM^{--}$. The distribution of $ReM^{--}$ is expected to be a Gaussian with a mean equal to the value of $ReM^{--}$ obtained from the source distribution, and the $\sigma$ of it will show the statistical uncertainties of the measured $ReM^{--}$.
As an example the distribution in Fig.\ref{fig:cos_phi_fit_illustration} is actually one of such generated distributions, and when we generated 10000 of such distributions, and fit with a $P\cdot cos(\phi)$ function, the resulting distribution of the $P$ is shown in
%%%%%%%%%%%%%%%%%%%%%%%%%%%%%%%% F I G U R E %%%%%%%%%%%%%%%%%%%%%%%%%%%%%%%%%%%%
\begin{figure}[!htb]
 \centering
 \grinp[width=0.75\tw]{img/Stat_Error_of_Fit.pdf}
 \caption{distribution of fit parameter $P$ obtained from 10000 generated distributions.}
 \label{fig:Stat_Error_of_P}
\end{figure}
%%%%%%%%%%%%%%%%%%%%%%%%%%%%%%%% F I G U R E %%%%%%%%%%%%%%%%%%%%%%%%%%%%%%%%%%%%
Fig.\ref{fig:Stat_Error_of_P}. The width of this distribution will represented the statistical uncertainty of $P$ which is related to $ReM^{--}$ by eq(\ref{eq:Re_M--}).

 
 \section{selection of kinematic points}
% \section{Extrapolating the interference term outside the acceptance region}
In this section we will discuss selection of TCS kinematic points, and compare TCS phase space to the DVCS phase space which is approved for Hall-C kinematics \cite{DVCS_HallC}.
%%%%%%%%%%%%%%%%%%%%%%%%%%%%%%%% F I G U R E %%%%%%%%%%%%%%%%%%%%%%%%%%%%%%%%%%%%
\begin{figure}[!htb]
 \centering
 \grinp[width=0.95\tw]{img/Kine1.pdf}
 \caption{The $Q^{2}$ vs $x_{B}$ distribution for the approved Hall-C kinematics (blue, green, and red areas), and TCS kinematics with CLAS12 acceptance (pink area). }
 \label{fig:kine1}
\end{figure}
%%%%%%%%%%%%%%%%%%%%%%%%%%%%%%%% F I G U R E %%%%%%%%%%%%%%%%%%%%%%%%%%%%%%%%%%%%
In TCS, the corresponding variables of $Q^{2} \equiv (p_{e}-p_{e}^{\prime 2})$ ans $x_{B} \equiv Q^{2}/2pq$, are $Q^{\prime 2} = M_{e^{-}e^{+}}^{2}$ and $\tau = Q^{\prime 2}/2pq$.
In Fig.\ref{fig:kine1} shown the $Q^{2}$ vs $x_{B}$ distribution for the approved Hall-C kinematics (blue, green, and red areas), and TCS kinematics with CLAS12 acceptance (pink area). 

In that plot the black solid line corresponds to the maximum achievable $Q^{2}$ for a given
beam energy. Usually in DVCS experiments maximum $Q^{2}$ is not accessible experimentally, since
it requires detection of a very low momentum electron, which are usually out of the acceptance, therefore as one can see DVCS kinematic coverage doesn't reach to the maximum accessible $Q^{2}$ values. In TCS however maximum $Q^{2}$ is well inside the the acceptance region for CLAS12 detector (both leptons are high energy and have large scattering angles).

In the fig.\ref{fig:kine1}, areas inside solid blue lines are proposed kinematic regions for TCS analysis.
In particular the lowest $x_{B}$ region ($x_{B}\in(0.2 - 0.3)$) can be complementary to DVCS,
since this $Q^{2}, x_{B}$ point is not accessible in any JLab DVCS experiments. Other points can serve as a test of universality of GPDs. I want to mention that these regions are selected in a way to have roughly equal statistics in all bins, and in total there was about 6000 expected events in each $Q^{\prime 2}, x_{B}$ bin.
 
 \section{Summary and outlook}
 It was demonstrated here that, the real part of the scattering amplitude $ReM^{--}$ can be extracted by fitting the WCRS as a function of $\phi$ angle. This is somewhat different from the calculation of the cosine moment which is proposed in paper. \cite{Berger:2001xd}. The procedure of estimation of statistical uncertainties is also discussed.
 
 The CLAS(12) acceptance imposes significant $\phi$ dependent variation of $\theta$ range, however because of the known $\theta$ depence of the cross section,  we can extrapolate the cross section to a fixed range in all $\phi$ bins, however proper determination of $\theta$ range needs more studies.
 
 Based on old FASTMC code and 120 days of full luminosity, this study chose 4 kinematic bins
 in $Q^{\prime 2}, x_{B}$ phase space, and the lowest $x_{B}$ can serve as a complementary to DVCS program since it has potential to cover a phase space not accessible in DVCS experiments at JLab. However situation now is different, and in order to estimate rates more realistically these studies should be repeated with GEMC and be reconstructed with the current CLAS12 reconstruction code, and we need to use the currently available luminosity to estimate expected statistics.

 
 
 \begin{thebibliography}{100}
  \bibitem{Berger:2001xd} 
  E.~R.~Berger, M.~Diehl and B.~Pire,
  %``Time - like Compton scattering: Exclusive photoproduction of lepton pairs,''
  Eur.\ Phys.\ J.\ C {\bf 23}, 675 (2002)
  doi:10.1007/s100520200917
  [hep-ph/0110062].
  %%CITATION = doi:10.1007/s100520200917;%%
  %111 citations counted in INSPIRE as of 27 Feb 2019
  
   \bibitem{Bochum_TCS} 
 Slides presented at thw workshop ``Deeply Virtual Compton Scattering: From Observables to GPDs'' 
 \href{http://www.tp2.rub.de/forschung/vortraege-und-workshops/dvcs2014/program/downloads/paremuzyan.pdf}{http://www.tp2.rub.de/forschung/vortraege-und-workshops/dvcs2014/program/downloads/paremuzyan.pdf}
 
 \bibitem{DVCS_HallC}
 DVCS Experiment proposal in Hall-C.
 \href{https://www.jlab.org/exp\_prog/proposals/13/PR12-13-010.pdf}{https://www.jlab.org/exp\_prog/proposals/13/PR12-13-010.pdf}
 

 \end{thebibliography}

\end{document}
